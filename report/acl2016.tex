%
% File acl2016.tex
%
%% Based on the style files for ACL-2015, with some improvements
%%  taken from the NAACL-2016 style
%% Based on the style files for ACL-2014, which were, in turn,
%% Based on the style files for ACL-2013, which were, in turn,
%% Based on the style files for ACL-2012, which were, in turn,
%% based on the style files for ACL-2011, which were, in turn, 
%% based on the style files for ACL-2010, which were, in turn, 
%% based on the style files for ACL-IJCNLP-2009, which were, in turn,
%% based on the style files for EACL-2009 and IJCNLP-2008...

%% Based on the style files for EACL 2006 by 
%%e.agirre@ehu.es or Sergi.Balari@uab.es
%% and that of ACL 08 by Joakim Nivre and Noah Smith

\documentclass[11pt]{article}
\usepackage{acl2016}
\usepackage{times}
\usepackage{url}
\usepackage{latexsym}
\usepackage{graphicx}
\usepackage{amsmath}
\usepackage{float}


%\aclfinalcopy % Uncomment this line for the final submission
%\def\aclpaperid{***} %  Enter the acl Paper ID here

%\setlength\titlebox{5cm}
% You can expand the titlebox if you need extra space
% to show all the authors. Please do not make the titlebox
% smaller than 5cm (the original size); we will check this
% in the camera-ready version and ask you to change it back.

\newcommand\BibTeX{B{\sc ib}\TeX}

\title{Character Identification on Multiparty Dialogue based on End-to-End Neural Coreference Resolution}

\author
{
   Pu-Chin Chen
  {\tt \{puchinchen@ucla.edu\}} \\
  Aoxuan Li 
  {\tt \{a811278305@gmail.com\}} \\
  Yutian Zhang
  {\tt \{yutianzh0527@gmail.com\}} \\
  Xin Liu
  {\tt \{xinliu627@ucla.edu\}} \\
}

\date{}

\begin{document}
\maketitle
\begin{abstract}
<<<<<<< HEAD
  Coreference resolution and entity linking are two important research topics in NLP, gaining growing attentions over years. In this project, we combined these two tasks together to accomplish a more complicated task, which is identifying the characters in multiparty dialogues. Different from previous traditional methods of coreference resolution, a state-of-the-art end-to-end model is adopted, which significantly outperforms other previous work. In addition, our entity-linking model is based on CNN instead of knowledge databases. Finally,we evaluated our models using the scripts from real TV shows.The results of coreference resolution are satisfactory, but the performance of entity linking does not live up to our expection. We analyzed the possible underlying reasons, which pointing out the research direction in future work.
=======
  Coreference resolution and entity linking are two important research topics in NLP, gaining growing attentions over years. In this project, we combined these two tasks together to accomplish a more complicated task, which is identifying the characters in multiparty dialoague. Different from previous traditional methods of coreference resolution, a state-of-the-art end-to-end model is adoptd, which significantly outpeforms other previous work.In addition, our entity-linking model is based on CNN instead of knowledge databases. Finally,we evaluated our models using the scripts from real TV shows.The results of coreference resolution are satisfactory, but the performance of entity linking does not live up to our expections.We analyzed the possible underlying reasons, which pointing out the research direction in future work.
>>>>>>> 57f72a367eea49fbab09890151d4683d9cea52c6
\end{abstract}

\section{Introduction}

Our main goal is to accomplish a shared task in SemEval 2018 - Task 4: Character Identification on Multiparty Dialogues. This tasks requires us to build a system which can identify different mentions in multiparty dialogues as corresponding characters in the show.This task is rather challenging, for cross-document entity resolution is imperative for identifying such mentions as real characters.

Literally, the character identification problem is tackled as a coreference resolution task with a further step on entity linking. In terms of this task, the baseline model generates mentions from a coreference system, and then each coreference chain is linked to a specific character identity. Both parts are implemented with agglomerative convolutional neural network in previous system (Chen et al., 2017; Chen and Choi, 2016).Alternatively, we intend to use Bidirectional-LSTM with attention mechanism to address the same problem, which proves to have a satisfying performance in many NLP tasks. In our project, we apply the end-to-end neural coreference resolution model introduced by Lee and He, etc. (Lee et al., 2017). After obtaining the predicted clusters of mentions by the end-to-end coreference system, we choose the same algorithm of entity-linking as the one Chen proposed in his paper.

\section{Related Work}
\subsection{Coreference resolution}
<<<<<<< HEAD
Machine learning methods have a long history in coreference resolution. However, the learning problem is challenging and, until very recently, hand-engineered systems built on top of automatically produced parse trees (Raghunathan et al., 2010) outperformed all learning approaches. Durrett and Klein (2013) showed that highly lexical learning approaches reverse this trend, and more recent neural models (Wiseman et al., 2016; Clark and Manning, 2016b,a) have achieved significant performance gains. However, all of these models still use parsers for head features and include highly engineered mention proposal algorithms. Such pipelined systems suffer from two major drawbacks: (1) parsing mistakes can introduce cascading errors and (2) many of the hand-engineered rules do not generalize to new languages or domains.  

The end-to-end coreference resolution model we used in this task significantly outperforms all previous work without using a syntactic parser or hand-engineered mention detector.

\subsection{Entity Linking}

Entity linking has traditionally relied heavily on knowledge databases, most notably, Wikipedia, for entities (Mihalcea and Csomai, 2007b; Ratinov et al., 2011b; Gattani et al., 2013; Francis-Landau et al., 2016). Although we do not make use of knowledge bases, our task is closely aligned to entity linking. Recent advances in entity linking are
also applicable to our task since we see Francis-Landau et al. (2016) use convolutional nets to capture semantic similarity between a mention and an entity by comparing context of the mention with the description of the entity. This work validates our usage of deep learning for character identification.
=======
The area of coreference resolution has resorted to the techniques of machine learning for many years. Still, the learning problem is challenging and, until very recently, hand-engineered systems built on top of automatically produced parse trees (Raghunathan et al., 2010) outperformed all learning approaches. Durrett and Klein (2013) showed that highly lexical learning approaches reverse this trend, and more recent neural models (Wiseman et al., 2016; Clark and Manning, 2016b,a) have achieved noticeable performance gains. However, all of the aforementioned models still use parsers for head features and include highly engineered mention proposal algorithms.Such pipelined systems suffer from two major drawbacks. First, parsing mistakes can introduce cascading errors. Second, many hand-engineered rules do not generalize ideally to new languages or domains.  

\subsection{Entity Linking}

Entity linking has traditionally relied heavily on knowledge databases, most notably, Wikipedia, for entities (Mihalcea and Csomai, 2007b; Ratinov et al., 2011b; Gattani et al., 2013; Francis-Landau et al., 2016).Although we do not make use of knowledge bases, our task is closely aligned to entity linking. Recent advances in entity linking are also applicable to our task since Francis-Landau et al. (2016) use CNN to capture semantic similarity between a mention and an entity by comparing context of the mention with the description of the entity. This work justifies our usage of deep learning for character identification.
>>>>>>> 57f72a367eea49fbab09890151d4683d9cea52c6

As indicated by the Dialogue State Tracking Challenges hosted by Microsoft (Kim et al., 2015), the dialogue tracking is an expanding task. The fact that an ongoing conversation can be dynamically tracked (Henderson et al., 2013) is exciting and applicable to our task because the state of a conversation may yield significant hints for entity linking and coreference resolution. Speaker identification, a task similar to character identification, has already shown some success with partial dialogue tracking by dynamically identifying speakers at each turn in a dialogue using conditional random field models.

\section{End-to-End Coreference Resolution}

<<<<<<< HEAD
\subsection{Introduction}
Recent coreference models usually rely on syntactic parsers. However, the end-to-end coreference resolution directly consider all the spans up to a maximum length in the document as potential mentions,  compute the probability of possible ancestors (previous span) for each span, and directly optimizes the marginal likelihood of antecedent spans from gold coreference clusters.

Since the number of potential mentions is very large, it is impractical to score all span pairs. So this model uses unary mention scores to prune the space of pairs of spans (spans and antecedents), which significantly reduce the pairwise computations.
=======
\subsection{Basic Idea}
Traditional coreference models usually rely on syntactic parsers from the perspective of the linguistics. In contrast, the end-to-end coreference resolution directly considers all the spans up to a maximum length in the document as potential mentions,  computes the probability of possible ancestors (previous span) for each span, and directly optimizes the marginal likelihood of antecedent spans from gold coreference clusters.

Since the number of potential mentions is very large, it is impractical to score all span pairs. Therefore, this model uses unary mention scores to prune the space of pairs of spans(spans and antecedents), which significantly reduce the pairwise computations.
>>>>>>> 57f72a367eea49fbab09890151d4683d9cea52c6

\subsection{Model}

\begin{figure}[h]
                
 \includegraphics[width=0.5\textwidth]{02.jpg}
 \caption{First step of end-to-end coreference resolution model}
                
\end{figure}

The first step is to compute embedding representations of spans for scoring potential entity mentions. A  bidirectional LSTM (Hochreiter and Schmidhuber, 1997) is applied to encode the lexical information of both the inside and outside of each span. The model also includes an head-finding attention mechanism over words in each span to model head words.
Low-scoring spans are pruned, so that only a manageable number of spans is considered for coreference decisions.

\begin{figure}[h]                
 \includegraphics[width=0.5\textwidth]{03.jpg}
 \caption{Second step of the model}             
\end{figure}

Second step is to compute antecedent scores from pairs of span representations. And then sum the mention scores of both spans and their pairwise antecedent score as the final coreference score of this pair of spans.
The antecedent scoring function includes explicit element-wise similarity of each span pair and a learned 20-dimensional feature vector which encoding all features like speaker genre, span distance, mention width.

\subsection{Learning}
During the learning period, we optimize the marginal log-likelihood of all correct antecedents implied by the gold clustering:

\begin{equation}
log\prod^{N}_{i=1}\sum_{\widehat{y}\in Y(i)\cap \text{GOLD}(i)}P(\widehat{y})
\end{equation}


where GOLD(i) is the set of spans in the gold cluster containing span i. If span i does not belong to a gold cluster or all gold antecedents have been pruned, GOLD(i) = \{$\varepsilon $\}.


\section{Entity Linking}

\subsection{Introduction}
Once the coreference resolution system is built and trained, we can predict co-references represented by clusters for each scene document. The next step should be making predictions from clusters to TV show character id. We model this process as an entity linking task. This section describes our entity linking model that takes the mention embeddings and the mention-pair embeddings generated by ACNN and classifies each mention to one of the character labels.

\subsection{Model}
\begin{figure}[h]                
 \includegraphics[width=0.5\textwidth]{05.jpg}
 \caption{Character identification model}             
\end{figure}

We prepare three embeddings generated by mentions which are predicted by precious co-reference system. The first embedding is mention embedding; the second is embedding of the cluster including the mention; the third is generated by mention-pair embedding, which pairs the mention with reaming mention in the same cluster. Here are the formal formula:
\begin{eqnarray}
\mathbf{R}_s(C_m) = [\mathbf{r}_s(m_1),\mathbf{r}_s(m_1),...,\mathbf{r}_s(m_{|C_m|})]\\
\mathbf{R}_p(C_m,m) = [\mathbf{r}_p(m_i,m)|m_i \neq m]
\end{eqnarray}



In order to fix the input tensor size of both cluster embedding and mention-pair embedding, we perform avg-pooling and max-pooling for both embeddings. Then each of pooling layers is passed to a convolutional layer. Finally, we concat the mention embedding, cluster CNN embedding and mention-pair CNN embedding.

\begin{equation}
\mathbf{r}_s(C_m) = \textrm{CONV}_s(\begin{bmatrix}
\text{avg\_pool}(\mathbf{R}_s(C_m))\\
\text{max\_pool}(\mathbf{R}_s(C_m))
\end{bmatrix}
\end{equation}


\begin{equation}
\mathbf{r}_p(C_m,m) = \textrm{CONV}_p(\begin{bmatrix}
\text{avg\_pool}(\mathbf{R}_s(C_m,m))\\
\text{max\_pool}(\mathbf{R}_s(C_m,m))
\end{bmatrix}
\end{equation}


After concatenation, we feed them into a feed-forward neural network with two hidden layers. Final output is a softmax with the idmention of entity id number in Friends TV show.

\subsection{Implementation}
Our training data has 374 documents, which contains totally 13280 gold mentions that have gold entity label. Firstly, we need to get gold clusters and generate three embeddings, i.e. mention, cluster and mention pair embeddings. Since all gold clusters are labeled with entity ids, we can flat clusters to mentions while computing three embeddings. We can easily get mention embeddings from the pretrained coref model.


\section{Dataset and Evaluation Metrics}
\subsection{Data Description}
The data comes from the scripts of the first two seasons of TV show Friends, which are already annotated for this task. Specifically, each season consists of episodes, and each episode is comprised of scenes. Furthermore, each scene is segmented into sentences. All datasets follow the CoNLL 2012 Shared Task data format. Each sentence is delimited by a new line and each column indicates the information regarding the word or the punctuation.
\begin{table*}[]
\centering
\label{my-label}
\begin{tabular}{|l|l|l|l|l|l|l|l|l|l|}
\hline
        & \multicolumn{3}{l|}{No Features} & \multicolumn{3}{l|}{No Heads} & \multicolumn{3}{l|}{Normal}   \\ \hline
        & F1       & Recall   & Precision  & F1      & Recall  & Precision & F1      & Recall  & Precision \\ \hline
MUC     & 79.14\%  & 76.98\%  & 81.42\%    & 80.90\% & 80.38\% & 81.43\%   & 80.84\% & 81.64\% & 80.71\%   \\ \hline
B\textsuperscript{3}      & 60.68\%  & 55.22\%  & 67.34\%    & 64.38\% & 62.54\% & 66.34\%   & 65.39\% & 68.89\% & 62.23\%   \\ \hline
CEAF\textsubscript{e} & 48.47\%  & 48.21\%  & 48.43\%    & 52.89\% & 50.54\% & 55.48\%   & 51.80\% & 53.51\% & 50.20\%   \\ \hline
AVG     & 62.76\%  & 60.14\%  & 65.73\%    & 66.06\% & 64.49\% & 67.75\%   & 66.02\% & 68.02\% & 64.17\%   \\ \hline
\end{tabular}
\caption{Coreference Resolution Performances of No Features, No Heads, and Normal}
\end{table*}
\subsection{Data Split}
Considering the amount of total data is not so sufficient, we simply split them into two datasets for different purposes, training and evaluation sets. The exact ratio of our training set to our evaluation set is 4:1. The evaluation set contains seventy-four scenes in \textit{Friends}.

\subsection{Coreference Evaluation Metrics}
The coreference results are evaluated with three metrics apropos coreference resolution:MUC, $B^{3}$,and $CEAF_e$.The precision, recall and F1 score of each metric approach are calculate separately and then the averages of them are obtained, which are utilized to compare against each other.

\textbf{MUC}\\
MUC(Vilain et al.,1995) concerns the number of pairwise links needed to be inserted or removed to map system responses to gold keys.The number of links shared by  system and gold are calculated. In addition, minimum numbers of links needed to describe coreference chains of the system and gold are computed as well.

\textbf{B\textsuperscript{3}} \\
Rather than evaluating coreference chains merely on their links,$B^{3}$(Bagga and Baldwin,1998) metric computes precision and recall on a mention level. System performance is evaluated by the average of all mentions scores.


\textbf{CEAF\textsubscript{e}} \\
$CEAF_e$(Luo,2005) metric further clarifies the downside of $B^{3}$, in which entities  can be used more than once during evaluation. Consequently, both multiple coreference chains of the same entity and chains with mentions of multiple entities are not penalized. To mitigate the aforementioned problem, $CEAF_e$ evaluates exclusively on the best one-to-one mapping between the system’s and gold’s entities.


\section{Results and Analysis}
\subsection{Ablations}
<<<<<<< HEAD
To show whether each component in our model is actually important, we ablate various parts of the architecture and report the average F1 on these experiments. We conducted one complete experiment and two ablation experiments, no heads and no features. In our dataset, the width of mention is always one or two words, the head-finding attention mechanism for each span seems to be useless, so we select no head ablation to verify this guess.  And we also want to show the significance of feature encoding.

\subsection{Coreference Resolution Results}
We conducted one complete experiment and two ablation experiments, no heads and no features. The results are shown in Table 1. Comparisons of those results shows no heads experiments has the best result. The reasons is, in Friends dataset, most mentions are single word, which means mention start and mention end are the same word and mention head has no significant contribution in this case. The result also indicates feature is an important factor of this model. In terms of coreference resolution, this model has an encouraging results comparing to most models.
=======
To verify whether each component in our model is actually indispensable, we ablate various parts of the architecture and report the average F1 values on these experiments. We conducted one complete experiment and two ablation experiments, no heads and no features. In our dataset, the width of mention is one or two most of the time owing to the nature of daily dialogue, so the head-finding attention mechanism for each span seems to be useless.We select no head ablation to verify this guess. Besides, we also intent to demonstrate the significance of feature encoding.

\subsection{Coreference Resolution Results}
We conducted one complete experiment and two ablation experiments, no heads and no features. The performances of coreference resolution using these three models are shown in Table 1. 
>>>>>>> 57f72a367eea49fbab09890151d4683d9cea52c6

As we expected, there is almost no difference between the F1 value of normal model and F1 value of no heads model. What's more, the results unveil that the feature encoding indeed contributes to the better performance of coreference system, for the F1 value decreases conspicuously after it is deprived from the complete system. 

The overall performance of coreference resolution is satisfactory and encouraging according to the F1 evaluation metric, which confirms the efficacy of the end-to-end system.

\subsection{Entity Linking Results}
We only got about 20\% accuracy on character identification prediction, addressing that simply using combination of mention embeddings generated from coreference system is weak for predcting entity ids. It fails to contain cluster information even though the coreference system does be trained to optimize the clustering accuracy. 

Another issue is that our models actually overfit. In the beginning, we used the ACNN model and overfit quite easily. Then we tried to reduce model complexities by only using one layer feed-forward neural network on mention embeddings, getting rid of cluster embedding and mention pair embedding. However, the preditcion precision is still low. This problem is arisen from two reasons. Firstly, this dataset is too small to use very complex model like deep neural networks. Secondly, consider our model, mention embeddings are extracted from merely word embeddings, which are insufficient to capture entity information.

\section{Conclusion and Future work}

In this work, we implemented a character identification system by integrating the neural coreference system with a CNN-based entity linking model. There are many open spaces for improvements. We should conduct the following experiments in the future:

{\bf Fortified Mention Embedding} 
The coreference model, proposed by Lee and He, does not add speaker ids and other features until calculating antecedent scores. We can add these features earlier in mention embeddings after LSTM so as to directly utilize the information regarding speakers.   

{\bf End-to-End Training}
Another improvement should be adding one layer for output entity ids after the coreference scores. The new loss function will be multiclass cross entropy. We can pre-train the coreference system and then train the entity linking part, by fixing the coreference model and optimize the loss. Inspired by (Clark and Manning, 2016), the further step will be designing loss function proposed in their work and boost the reliability of our system using reinforcement learning.



% include your own bib file like this:
%\bibliographystyle{acl}
%\bibliography{acl2016}
\bibliography{acl2016}
\bibliographystyle{acl2016}

\appendix





\end{document}


